\documentclass[11pt]{article}
\usepackage{colacl}
\sloppy



\title{Waht kinda typoz do poeple mak?}
\author
{Anonymous}



\begin{document}
\maketitle


\section{Introduction}

\paragraph{} The goal of this report is to determine what kind of typographical errors people make. In this report, one baseline algorithm and two advanced algorithms will be implemented for comparisons and evaluations.

\subsection{Dataset}

\paragraph{} The dataset used in this report involves 4453 common misspelling errors made by the editors of Wikipedia, and their corresponding truly intended spellings.

\begin{table}[h]
 \begin{center}
\begin{tabular}{| l | l | l |}

      \hline
      Evaluation Metric & 30\% & 100\% \\
      \hline\hline
      Precision & NA & 0.2604 \\
      Recall & NA & 0.7905 \\
      \hline

\end{tabular}
\caption{Compare 30\% and 100\% of dataset}\label{table1}
 \end{center}
\end{table}

According to the evaluation metrics of the baseline algorithm shown above, there is no much difference between 30\% and 100\% of the dataset. Therefore, only 30\% random selected tokens of the dataset (1335 tokens) will be used for the rest of the algorithms.

\subsection{Previous Work}

\section{Hypothesis}

Text.

\section{Method}


\subsection{Global Edit Distance (GED)}

\subsubsection{Parameters}

Levenshtein Distance.

\begin{table}[h]
 \begin{center}
\begin{tabular}{| l | l |}

      \hline
      Evaluation Metric & Levenshtein \\
      \hline\hline
      Precision & 0.2604 \\
      Recall & 0.7905 \\
      F-Score & 0.3918 \\
      \hline

\end{tabular}
\caption{Compare 30\% and 100\% of dataset}\label{table1}
 \end{center}
\end{table}

\subsubsection{Implementation}



 
\subsection{Soundex}

\section{Evaluation}

\paragraph{} In this report, the algorithms applied will give multiple predictions for each misspelled word, therefore, precision and recall should be used as the evaluation metrics.

In order to compare between the baseline and advanced algorithms, precision and recall can be combined into a single evaluation metric, called F-Score, which is the harmonic mean of precision and recall. [citation]

\section{Discussion}


\section{Conclusion}

Concluding text.

\bibliographystyle{acl}
\bibliography{mybib}

\end{document}